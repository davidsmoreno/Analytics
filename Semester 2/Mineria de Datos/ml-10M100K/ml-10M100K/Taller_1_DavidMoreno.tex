% Options for packages loaded elsewhere
\PassOptionsToPackage{unicode}{hyperref}
\PassOptionsToPackage{hyphens}{url}
%
\documentclass[
]{article}
\usepackage{amsmath,amssymb}
\usepackage{iftex}
\ifPDFTeX
  \usepackage[T1]{fontenc}
  \usepackage[utf8]{inputenc}
  \usepackage{textcomp} % provide euro and other symbols
\else % if luatex or xetex
  \usepackage{unicode-math} % this also loads fontspec
  \defaultfontfeatures{Scale=MatchLowercase}
  \defaultfontfeatures[\rmfamily]{Ligatures=TeX,Scale=1}
\fi
\usepackage{lmodern}
\ifPDFTeX\else
  % xetex/luatex font selection
\fi
% Use upquote if available, for straight quotes in verbatim environments
\IfFileExists{upquote.sty}{\usepackage{upquote}}{}
\IfFileExists{microtype.sty}{% use microtype if available
  \usepackage[]{microtype}
  \UseMicrotypeSet[protrusion]{basicmath} % disable protrusion for tt fonts
}{}
\makeatletter
\@ifundefined{KOMAClassName}{% if non-KOMA class
  \IfFileExists{parskip.sty}{%
    \usepackage{parskip}
  }{% else
    \setlength{\parindent}{0pt}
    \setlength{\parskip}{6pt plus 2pt minus 1pt}}
}{% if KOMA class
  \KOMAoptions{parskip=half}}
\makeatother
\usepackage{xcolor}
\usepackage[margin=1in]{geometry}
\usepackage{color}
\usepackage{fancyvrb}
\newcommand{\VerbBar}{|}
\newcommand{\VERB}{\Verb[commandchars=\\\{\}]}
\DefineVerbatimEnvironment{Highlighting}{Verbatim}{commandchars=\\\{\}}
% Add ',fontsize=\small' for more characters per line
\usepackage{framed}
\definecolor{shadecolor}{RGB}{248,248,248}
\newenvironment{Shaded}{\begin{snugshade}}{\end{snugshade}}
\newcommand{\AlertTok}[1]{\textcolor[rgb]{0.94,0.16,0.16}{#1}}
\newcommand{\AnnotationTok}[1]{\textcolor[rgb]{0.56,0.35,0.01}{\textbf{\textit{#1}}}}
\newcommand{\AttributeTok}[1]{\textcolor[rgb]{0.13,0.29,0.53}{#1}}
\newcommand{\BaseNTok}[1]{\textcolor[rgb]{0.00,0.00,0.81}{#1}}
\newcommand{\BuiltInTok}[1]{#1}
\newcommand{\CharTok}[1]{\textcolor[rgb]{0.31,0.60,0.02}{#1}}
\newcommand{\CommentTok}[1]{\textcolor[rgb]{0.56,0.35,0.01}{\textit{#1}}}
\newcommand{\CommentVarTok}[1]{\textcolor[rgb]{0.56,0.35,0.01}{\textbf{\textit{#1}}}}
\newcommand{\ConstantTok}[1]{\textcolor[rgb]{0.56,0.35,0.01}{#1}}
\newcommand{\ControlFlowTok}[1]{\textcolor[rgb]{0.13,0.29,0.53}{\textbf{#1}}}
\newcommand{\DataTypeTok}[1]{\textcolor[rgb]{0.13,0.29,0.53}{#1}}
\newcommand{\DecValTok}[1]{\textcolor[rgb]{0.00,0.00,0.81}{#1}}
\newcommand{\DocumentationTok}[1]{\textcolor[rgb]{0.56,0.35,0.01}{\textbf{\textit{#1}}}}
\newcommand{\ErrorTok}[1]{\textcolor[rgb]{0.64,0.00,0.00}{\textbf{#1}}}
\newcommand{\ExtensionTok}[1]{#1}
\newcommand{\FloatTok}[1]{\textcolor[rgb]{0.00,0.00,0.81}{#1}}
\newcommand{\FunctionTok}[1]{\textcolor[rgb]{0.13,0.29,0.53}{\textbf{#1}}}
\newcommand{\ImportTok}[1]{#1}
\newcommand{\InformationTok}[1]{\textcolor[rgb]{0.56,0.35,0.01}{\textbf{\textit{#1}}}}
\newcommand{\KeywordTok}[1]{\textcolor[rgb]{0.13,0.29,0.53}{\textbf{#1}}}
\newcommand{\NormalTok}[1]{#1}
\newcommand{\OperatorTok}[1]{\textcolor[rgb]{0.81,0.36,0.00}{\textbf{#1}}}
\newcommand{\OtherTok}[1]{\textcolor[rgb]{0.56,0.35,0.01}{#1}}
\newcommand{\PreprocessorTok}[1]{\textcolor[rgb]{0.56,0.35,0.01}{\textit{#1}}}
\newcommand{\RegionMarkerTok}[1]{#1}
\newcommand{\SpecialCharTok}[1]{\textcolor[rgb]{0.81,0.36,0.00}{\textbf{#1}}}
\newcommand{\SpecialStringTok}[1]{\textcolor[rgb]{0.31,0.60,0.02}{#1}}
\newcommand{\StringTok}[1]{\textcolor[rgb]{0.31,0.60,0.02}{#1}}
\newcommand{\VariableTok}[1]{\textcolor[rgb]{0.00,0.00,0.00}{#1}}
\newcommand{\VerbatimStringTok}[1]{\textcolor[rgb]{0.31,0.60,0.02}{#1}}
\newcommand{\WarningTok}[1]{\textcolor[rgb]{0.56,0.35,0.01}{\textbf{\textit{#1}}}}
\usepackage{graphicx}
\makeatletter
\def\maxwidth{\ifdim\Gin@nat@width>\linewidth\linewidth\else\Gin@nat@width\fi}
\def\maxheight{\ifdim\Gin@nat@height>\textheight\textheight\else\Gin@nat@height\fi}
\makeatother
% Scale images if necessary, so that they will not overflow the page
% margins by default, and it is still possible to overwrite the defaults
% using explicit options in \includegraphics[width, height, ...]{}
\setkeys{Gin}{width=\maxwidth,height=\maxheight,keepaspectratio}
% Set default figure placement to htbp
\makeatletter
\def\fps@figure{htbp}
\makeatother
\setlength{\emergencystretch}{3em} % prevent overfull lines
\providecommand{\tightlist}{%
  \setlength{\itemsep}{0pt}\setlength{\parskip}{0pt}}
\setcounter{secnumdepth}{-\maxdimen} % remove section numbering
\ifLuaTeX
  \usepackage{selnolig}  % disable illegal ligatures
\fi
\usepackage{bookmark}
\IfFileExists{xurl.sty}{\usepackage{xurl}}{} % add URL line breaks if available
\urlstyle{same}
\hypersetup{
  pdftitle={Taller 1 Mineria de Datos},
  pdfauthor={David Morneo},
  hidelinks,
  pdfcreator={LaTeX via pandoc}}

\title{Taller 1 Mineria de Datos}
\author{David Morneo}
\date{February 08, 2024}

\begin{document}
\maketitle

\subsubsection{Punto 1}\label{punto-1}

\emph{Pregunta}: Identifique qué procesamiento de datos fue tenido en
cuenta. Relacione las funciones de R implementadas y explique su
funcionamiento de forma concreta.

\emph{Respuesta}: Primero observamos que los datos leidos son del tipo
.dat. Lo que hace el código en la función \textbf{gsub()} es sustituir
el string ``::'' por un espacio. La función \textbf{str\_split\_fixed()}
realiza un split de cada elemento del vector anterior, indicando al
final con un 3 para que cree 3 columnas. Luego, se asignan nombres a las
columnas utilizando \textbf{colnames(dataframe)\textless-c()} para
definir el nombre de las columnas.

Posteriormente, realiza un merge de los datos de ratings y movies
basándose en el \emph{movieId}, que ambas bases de datos comparten, y
une los datos mediante un Left Join, conservando los ratings como la
tabla de la izquierda. Depués, divide los datos en conjuntos de Test y
Validación utilizando la función \textbf{createDataPartition()}, la cual
recibe la variable respuesta, y el número de veces que se quiere
dividir(este número puede variar si se utiliza validación curzada) y
\(p\), el porcentaje de los datos que se utilizarán para validación.

\begin{Shaded}
\begin{Highlighting}[]
\NormalTok{validation }\OtherTok{\textless{}{-}}\NormalTok{ temp }\SpecialCharTok{\%\textgreater{}\%} 
     \FunctionTok{semi\_join}\NormalTok{(edx, }\AttributeTok{by =} \StringTok{"movieId"}\NormalTok{) }\SpecialCharTok{\%\textgreater{}\%}
     \FunctionTok{semi\_join}\NormalTok{(edx, }\AttributeTok{by =} \StringTok{"userId"}\NormalTok{)}
\end{Highlighting}
\end{Shaded}

La función \%\textgreater\% representa un pipeline, que permite llamar a
otras funciones secuencialmente. Esto es útil para realizar varias
transformaciones en los datos en un solo paso.

\subsubsection{Punto 2}\label{punto-2}

\emph{Pregunta}: Explique cuáles análisis exploratorios hicieron por
cada variable, ¿qué beneficio le trajo al modelo final? ¿Cuáles son
algunas funciones clave en ese sentido?

\emph{Respuesta}:

En el análisis exploratorio, se calculo el número de usuarios, de
peliculas, de generos y se análizo el intervalo de fechas de nuestros
datos, se hizo esto para los datos de validación y de entrenamiento.

\textbf{Análisis Exploratorio Rating}: Para la variable Rating, se
realiza un resumen descriptivo, incluyendo la media y la desviación
estándar.

Luego, se crea una gráfica de barras contando el número de ratings por
cada 0.5.

Posteriormente, se elabora la gráfica ordenada de mayor a menor. Una de
las funciones clave en estos gráficos es \textbf{ggplot(data=DataFrame,
aes(x, y)) + Geom(additional parameters)}. Esta función recibe el
conjunto de datos; la función aes() define el mapeo entre los datos (se
elige qué se va a graficar); y el tercer parámetro se añade para
especificar qué tipo de gráfico se desea crear, ya sea de línea o de
barras, además de los demás parámetros que agregan el título, nombres y
demás.

Otra función que me parecio interesante fue
\textbf{mutate(group=cut(n\_rating\_of\_movie,breaks=c(-Inf,mean(n\_rating\_of\_movie),Inf)}
Lo que hace \textbf{mutate()} lo que nos permite es agregar variables a
un dataframe of modificar algunas que ya sean existentes y la variable
\textbf{cut()} se utiliza para dividir el rango de una variable continua
en intervalos. luego cuando utilizamos **breaks=c(-Inf,mean(),Inf), le
definimos los intervalos. Como sabemos en R un número Inf es el
equivalente a 2\^{}31.

\textbf{Análisis Exploratio Movie}: Para el análisis de las peliculas
una de las funciones importantes es \textbf{group\_by(movieId)} la cual
como su nombre lo indica agrupa filas en este caso por la media y la
desviación estandar, puero puede utilizarce para suma, media, y otras
funciones más para organizar mejor los datos y hacer gráficos más
especificos.

\begin{Shaded}
\begin{Highlighting}[]
\NormalTok{movie\_sum }\OtherTok{\textless{}{-}}\NormalTok{ edx }\SpecialCharTok{\%\textgreater{}\%} \FunctionTok{group\_by}\NormalTok{(movieId) }\SpecialCharTok{\%\textgreater{}\%}
  \FunctionTok{summarize}\NormalTok{(}\AttributeTok{n\_rating\_of\_movie =} \FunctionTok{n}\NormalTok{(), }
            \AttributeTok{mu\_movie =} \FunctionTok{mean}\NormalTok{(rating),}
            \AttributeTok{sd\_movie =} \FunctionTok{sd}\NormalTok{(rating))}
\end{Highlighting}
\end{Shaded}

Una función crucial en el análisis de datos con R es
\textbf{summarize()}, que se emplea para generar resúmenes de los datos.
En el contexto mencionado, esta función se utiliza para calcular
estadísticas clave como la media y la desviación estándar, organizando
los datos por \emph{MovieId}.

Un análisis visual particularmente revelador es la exploración de la
densidad del número de calificaciones. Este enfoque permite identificar
cómo se distribuyen predominantemente las calificaciones en términos
porcentuales.

La visualización gráfica es instrumental en este proceso, ya que
facilita la identificación de valores atípicos (outliers) que agregan
significado y profundidad a nuestra comprensión de los datos.

Dentro de las herramientas gráficas que encontramos útiles están
\textbf{geom\_hist()}, \textbf{geom\_point()}, y \textbf{geom\_vline()}.
Este último, por ejemplo, se utiliza para añadir una línea vertical en
el valor medio de las calificaciones por película, mediante el uso de
\textbf{aes(xintercept = mean(n\_rating\_of\_movie))}.

\subsubsection{Punto 3}\label{punto-3}

\emph{Pregunta}: Identifique la estructura de visualización utilizada.
¿Cuáles funciones fueron implementadas y cuál es su funcionalidad de
forma concreta?

\emph{Respuesta}: En el documento, detallo las funciones clave para la
creación de gráficos.

\subsubsection{Punto 4}\label{punto-4}

\emph{Pregunta}: ¿Qué le llamó la atención de esta aproximación? ¿Qué
hubiera hecho diferente?

\emph{Respuesta}: En la fase de análisis exploratorio, la estrategia de
segmentar las calificaciones en dos grupos, basándose en el promedio,
fue fundamental para el reconocimiento de valores atípicos y su posible
influencia en el modelo. Esta división revela variaciones en las
preferencias o sesgos de los usuarios.

Me resultó particularmente revelador cómo se mejoró la precisión del
modelo mediante la incorporación de un efecto individual por usuario
(bjbj\hspace{0pt}), ajustando las predicciones conforme a si un usuario
tiende a calificar por encima o por debajo del promedio, utilizando la
fórmula \(b_j = \bar{r}_j - \bar{r}\)

Una estrategia alternativa que contemplaría implica analizar las
correlaciones entre variables dentro de categorías determinadas para
destacar películas altamente calificadas. Luego procedería a realizar un
análisis de agrupación (clustering) con dichas variables para
identificar posibles categorías distintivas. Si se descubren categorías
relevantes, las incorporaría al modelo con el objetivo de mejorar la
precisión de las predicciones. Adicionalmente, examinaría los valores
atípicos y consideraría tácticas para atenuar su influencia, como la
aplicación de técnicas de palanca en la regresión lineal.

\end{document}
