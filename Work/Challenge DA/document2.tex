\documentclass{beamer}
\usetheme{metropolis}
\usepackage[utf8]{inputenc}
\usepackage[T1]{fontenc}
\usepackage{graphicx}
\usepackage{enumitem}

\setbeamertemplate{itemize item}{$\bullet$}
\setlist[itemize]{label=$\bullet$}


\title{Data Analysis for Marketing}
\author{David Moreno}
\date{\today}

\begin{document}
	
	\maketitle
	
\section{Data Exploration and Understanding}
\begin{frame}{Data Exploration and Understanding}
	\begin{itemize}
		\item In this stage, we delve into the dataset to grasp the meaning of each column.
		\item Descriptive statistics of each column, such as mean, min, max, and percentiles, are calculated. To see how are data is distributed.
		\item Visualizations are employed to uncover data distributions and to understand better each variable.
		
	\end{itemize}
\end{frame}
	
	\section{Data Cleaning}
	\begin{frame}{Data Cleaning}
		\begin{itemize}
			\item We delete all the duplicated, we look for NAN, and find Outliers that could potentially affect our data.

		\end{itemize}
	\end{frame}
	
	\section{Key Performance Indicators (KPIs)}
	\begin{frame}{Key Performance Indicators (KPIs)}
		\begin{itemize}
			\item Identify relevant KPIs aligned with campaign objectives.
			\item Quantitatively measure the impact of the campaign.
			\item Make Graphs using this KPIS and find the apropiate way of showing the graph.
		\end{itemize}
	\end{frame}
	
	
	\section{KPI Creation}
	\begin{frame}{KPI Creation}
		\begin{itemize}
			\item I create a KPIS,Click-Through Rate (CTR), a widely used marketing metric.
			\item This Kpi measures Engagement, it indicates whether your audience is interested enough to take action(or to do click in the add).
			\begin{align*}
				\text{CTR} & = \left(\frac{\text{Number of Clicks}}{\text{Number of Impressions}}\right) \times 100
			\end{align*}
		\end{itemize}
	\end{frame}
	
	\section{Data Visualization}
	\begin{frame}{Data Visualization}
		\begin{itemize}
			\item Select appropriate graphs to represent the data effectively.
			\item Emphasize visualization techniques that highlight critical insights for this dataset and campaign.
		\end{itemize}
	\end{frame}

\section{Data Analisis}
\begin{frame}
	\begin{itemize}
		\item Lets see how the CTR is moving between months
		
	\end{itemize}
\end{frame}
	
	
\end{document}