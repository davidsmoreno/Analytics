\documentclass{report}

\input{preamble}
\input{macros}
\input{letterfonts}

\title{\Huge{Modelación y Mejora de Procesos}\\Clase1}
\author{\huge{David Moreno}}
\date{\today}

\begin{document}

\maketitle

\section{Introducción a la optimización}

En Analytics hay 3 fuentes principales, el qué paso‚el qué va a pasar
y el cómo hago para que algo pase? Si definimos las tres categorias
las vemos asi. Los datos que tenemos que nos dicen: Estadistica
descriptiva, el que va a pasar nos los dice los modelos despriptivos y
el Cómo hago para que algo pase? Digamos que la acción es el proceso
de la optimización.

\begin{exa}
\textbf{Marketing mix:} En los últimos meses,una compania ha notado
una caída en la participación de mercado de su producto líder por lo
que ha tomado la decisión de cambiar su estrategia de mercadeo. Desde
las directivas de la compania se ha decidido destinar un presupuesto
mayor en el área de publicidad con el objetivo de aumentar las
ventas.\\ 
De acuerdo a las estimaciones de la companialos dos canales de
publicidad que generan mayor incremento en ventas son los anuncios
publicitarios televisivos y FacebookAds, el incremento seŕa de 2 M.\\

El tiempo disponible en televisión es de 360 Segundos por mes debido a
la disponibilidad de la televisora y las estimaciones del alcance
máximo de los anuncios de Facebook es de 6000 clicks por mes debido a
la segmentación realizada.\\

El presupuesto con el que cuuenta la compania es de 15 m.
El costo asociado a la compra de 30 segundos de tiempo televisivo
corresponde a 1m y el costo asociado a 500 clicks en FacebookAds
corresponde a 1M.\\

Ademas, en línea con sus objetivos estratégicos, se ha establecido que
por lo menos el 40 porciento de la inversión en esta nueva estrategia
debe hacerse en Facebook Ads.\\

Entonces cuánto debería invertir mensualmente la compania en cada uno
de los canales para obtener los mejores resultados en ventas?


\end{exa}




\end{document}
