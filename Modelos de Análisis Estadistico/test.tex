\documentclass{article}

\usepackage[spanish]{babel}
\usepackage{amsfonts}
\usepackage{amsmath}
\usepackage{amsthm}
\usepackage{geometry}

\geometry{
 a4paper,
 left=20mm,
 top=20mm,
 right=20mm,
 bottom=20mm
 }

\newtheorem{thm}{Teorema}[section]
\newtheorem{lem}[thm]{Lema}
\newtheorem{prop}[thm]{Proposición}



\theoremstyle{definition}
\newtheorem{defi}[thm]{Definición}
\newtheorem{exa}[thm]{Ejemplo}

\author{David Moreno}
\title{Lecure Week 1 }

\begin{document}
\maketitle
\section*{Introducción al pensamiento estadístico}

La intuición humana a menudo intenta responder a las mismas preguntas
que podemos responder utilizando el pensamiento estadístico, pero a
menudo se equivoca en la respuesta.
Por ejemplo, en a


\end{document}
