\documentclass{article}

\usepackage[spanish]{babel}
\usepackage{amsfonts}
\usepackage{amsmath}
\usepackage{amsthm}
\usepackage{geometry}

\geometry{
 a4paper,
 left=20mm,
 top=20mm,
 right=20mm,
 bottom=20mm
 }

\newtheorem{thm}{Teorema}[section]
\newtheorem{lem}[thm]{Lema}
\newtheorem{prop}[thm]{Proposición}



\theoremstyle{definition}
\newtheorem{defi}[thm]{Definición}
\newtheorem{exa}[thm]{Ejemplo}

\author{Juan Sebastian Gaitan Escarpeta}
\title{Introducción a los Espacios $L^P$, $L^\infty$ y espacios de
Hölder}

\begin{document}
\maketitle
\section*{Integral de Lebesgue}

La forma apropiada de introducir los espacios $L^p$, requere definir
una noción de integral de Lebesgue. Como el objetivo de este artículo
introductorio no es profundizar en el tema de de teoría de la medida
ni la integral de Lebesgue, no nos concentraremos en el tema, y el
lector podría (preeliminarmente) pensar todas las integrales que
aparecen a continuación como integrales de Riemann, en todo caso, si
$f$ es integrable en e lintervalo $[a,b]$ en el sentido de Riemann,
entonces también lo es en el sentido de Lebesgue, mas aún, ambas
integrales coinciden, es decir:
\[
\int_{[a,b]}^{} f(x) \text{d}x=\int_{a}^{b} f(x) \text{d}x.
\]

La razón precisa por la cual no se puede simplemente usar la integral
de Riemann, será explicada con precisión en la sección
\ref{sub:Espacio L1}.

\section{Espacios $L^P$}
\subsection{Espacio $L^1$}%
\label{sub:Espacio L1}

\begin{defi}
El conjunto $\left\{f  \colon X  \to \mathbb{R} : \int_{X}^{}
|f(x)| \, \text{d}x < \infty \right\}$ se denota por
$\mathfrak{L}^1(X)$.
\end{defi}

Podemos definir una relación de equivalencia sobre este conjunto definida por $f\sim g$ si
y sólo si $\int_{X}^{} |f-g| \text{d}x=0$. Si $X$ es un intervalo real
y usamos la integral de Riemann, esta relación de equivalencia implica
que $f$ y $g$ son iguales salvo en un \textit{conjunto de medida 0}.
La siguiente proposición tiene como objetivo dar luces a la naturaleza
de esta relación.

\begin{prop}
Sean $f,g \in \mathfrak{L}^1([a,b])$ funciones continuas, si $f\sim
g$, entonces $f=g$.
\end{prop}
\begin{proof}
	Demostraremos la contrarecíproca. Suponga que $f\neq g$, entonces
existe $x_0\in [a,b]$ tal que $f(x_0)\neq g(x_0)$. Esto significa que
 $|f(x_0)-g(x_0)|\neq 0$. Como $|f-g|$ también es continua, existe un
intervalo $[c,d]\subseteq [a,b]$ tal que para todo $x\in [c,d]$,
$|f(x)-g(x)|>0$, lo que significa que $\int_{[c,d]} |f(x)-g(x)|
\text{d}x>0$, implicando finalmente que $\int_{[a,b]} |f(x)-g(x)|
\text{d}x >0$.
\end{proof}

Si en funciones continuas, la relación $\sim$ es exactamente igual a
una igualdad, el lector se podría preguntar si esto es así en general.
Veamos un ejemplo:

\begin{exa}
Considere las funciones $f,g \in \mathfrak{L}^1([0,1])$ dadas por
\[
f(x)=
\begin{cases}
	1 \text{ si } x=0\\
  x^2 \text{ en otro caso }.
\end{cases}
\]
\[
g(x)=x^2
\]
Estas dos funciones son distintas puesto que sus valores son distintos
en $1$, pero es posible verificar que $f\sim g$. La clave es notar que
estas funciones son \textit{casi} iguales, lo que se puede formalizar
en terminos de que las funciones coinciden salvo en un conjunto de
medida $0$.
\end{exa}

\begin{defi}
El conjunto $L^1$ se define como el conjunto de clases de equivalencia
de $\mathfrak{L}^1$ bajo la relación $\sim$.
\end{defi}

Tomando las funciones del ejemplo anterior, concluimos que en $L^1$,
$[f]=[g]$, aunque usualmente no se usa la notación $[\cdot ]$ y
sólo escribimos $f=g$.

Consideremos por un momentos las funciones $f,g \colon [0,1] \to
\mathbb{R}$ dadas por $f(x)=0$ y 
\[
g(x)=
\begin{cases}
1 \text{ si } x\in \mathbb{Q} \\
0 \text{ en otro caso.}
\end{cases}
\]
Estas dos funciones, son \textit{casi} iguales, es decir el conjunto
en donde difieren ($\mathbb{Q}$) tiene medida cero, luego quisieramos
$f\sim g$, pero $g$ no es integrable en el sentido de Riemann, luego
$ \int_{0}^{1} g(x) \text{d}x$ no está definido, mientras que usando
la integral de Lebesgue, tenemos $ \int_{[0,1]}^{} g(x) \text{d}x=0$,
y efectivamente $f\sim g$

\begin{defi}
Para $f\in L^1(X)$, definimos $|f|_{L^1}=\int_{X}^{} |f(x)| \text{d}x$.
\end{defi}

Esta definción clarifica la necesidad de la relación $\sim$, puesto
que para que $L^1$ sea un espacio normado, se debe cumplir que
$|f|_{L^1}=0$ implica $f=0$, lo que sólo es cierto después de tomar
las clases de equivalencia de $\sim$. Más adelante, con una
demostración más general, veremos que $L^p$ es un espacio vectorial
normado.

\subsection{Espacio $L^2$}%
\label{sub:EspacioL^2}

\begin{defi}
Una función  $f  \colon X \to \mathbb{R}$, se dice \textit{cuadrado
integrable}, si $ \int_{X}^{} |f(x)|^2 \text{d}x<\infty$.
\end{defi}

De manera similar a la sección anterior, definimos $\mathfrak{L}^2(X)$
como el conjunto de todas las funciones $ f \colon X \to \mathbb{R} $
cuadrado integrables, y definimos la misma relación de equivalencia $\sim$
sobre $\mathfrak{L}^2(X)$ como $f\sim_2 g$ si $\int_{{X}}^{{}} {|f-g|}
\: d{x} {}$ =0. Si el lector está familiarizado con la integral de
Lebesgue, puede verificar que efectivamente $\sim_2$ al igual que
$\sim$ es una relación de equivalencia.
\begin{defi}
Definimos $L^2$ como el conjunto definido por las clases de
equivalencia de $\mathfrak{L}^2$ bajo la relación $\sim$.
\end{defi}

De manera similar a la sección anterior, podemos definir una norma en $L^2$

\begin{defi}
Para $f \in L^2$ definimos $|f|_{L^2}={\left(\int_{X}^{} |f(x)|^2
\text{d}x\right)}^{1/2}$.
\end{defi}

La peculiaridad de $L^2$, es que, además de ser un espacio vectorial
normado, también tiene un producto interno compatible con esta norma.

\begin{defi}
Para dos funciones $f,g\in L^2(X)$, definimos 
\[
\left<f,g \right> =\int_{X}^{} f(x)g(x) \text{d}x.
\]
\end{defi}
El lector puede verificar que $\left< \cdot, \cdot \right>$ es
efectivamente un producto interno, y que $\left<f,f \right> =
|f(x)|_{L^2}^{{2}}$.

\begin{exa}
Considere las funciones $f(x)=\sin x$ y $g(x)=x^2$ en el conjunto
$L^2([0,1])$. Entonces tenemos: $\left< f , g \right> =
\int_{[0,1]}^{} x^2 \sin x \, \text{d}x=
 -2 + \cos{\left(1 \right)} + 2 \sin{\left(1
\right)}=0.223244275483933 $. Además, $|f|_{L^2}= \left(\int_{[0,1]}^{}
{\left(x^2\right)}^2 \text{d}x \right) ^{\frac{1}{2}}=
0.447213595499958$.
\end{exa}

Como nota adicional, los espacios con producto interno, tales como
$L^2$ se conocen como espacios de Hilbert.

\subsection{Espacios $L^p$}%
\label{sub:EspaciosL^p}

\begin{defi}
Para cualquier $p\in (1,\infty)$, definimos el conjunto $\mathfrak{L}^p=\left\{ f \colon X \to
\mathbb{R} : \int_{X}^{} {|f(x)|}^p \text{d}x<\infty \right\}$, junto
con la relación de equivalencia $f\sim g$ si $\int_{X}^{}
{|f(x)-g(x)|} \text{d}x=0$. Finalmente definimos el conjunto $L^p$
como el conjunto de clases de equivalencia de la relación $\sim$.
\end{defi}

\begin{thm}{(Desigualdad de Minkowski)}
Sean $f,g\in L^P(X)$, entonces

\[
|f+g|_{L^p}\le |f|_{L^p}+|g|_{L^p}.
\]
\end{thm}

La demostración de la desigualdad de Minkowski está fuera del alcance
y el propósito de esta introducción, sin embargo este resultado es
supremamente importante dado que es la desigualdad triangular de la
$p$-norma. Salvo por este resultado, el lector debería ser capaz de
demostrar que la $p$-norma satisface todas las condiciones para ser
una norma de $L^p$. Un resultado mas débil se demostrará a
continuación para convencer al lector que $L^p$ efectivamente es un
espacio vectorial.

\begin{lem}
Sean $s,t\ge 0$ y $p>1$, entonces $(s+t)^p\le 2^{p-1}(s^p+t^p)$.
\end{lem}
\begin{proof}
	Note que la función $f(x)=x^p$ es convexa en los positivos, por lo
que, de la definición de convexidad con $t=\frac{q}{2}$, tenemos:
\begin{equation*}
\begin{split}
{\left( \frac{1}{2}s+ \frac{1}{2}t \right)}^p &\le
\frac{1}{2}s^p+\frac{1}{2}t^p\\
2^p{\left( \frac{1}{2}s+ \frac{1}{2}t \right)}^p &\le
2^p{\left(\frac{1}{2}s^p+\frac{1}{2}t^p \right)}\\
(s+t)^p &\le  2^p(s^p+t^p).
\end{split}
\end{equation*}
\end{proof}


\begin{proof}
	Usando el lema anterior y la desigualdad triangular, tenemos:
\[
\int_{X}^{} |f+g|^p \text{d}
\]
\end{proof}


\end{document}
